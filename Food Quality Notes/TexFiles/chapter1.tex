\chapter{Information of the course}

\section{Course content}

The course will introduce the concepts of food quality management and control from two different/complementary perspectives: 

1: a detailed reading of the ISO 22000 standard, and insight into the use of this standard, including HACCP (Hazard Analysis and Critical Control Points), for the management of food safety. 

2: monitoring of the food quality based on process monitoring tools such as Statistical Quality Control (SQC) and Lean Six Sigma to ensure a food product with minimal variation.

Producing food requires understanding of all process steps, and knowledge of the performance of each step is critical to estimate the variability of the final product. Knowing the variability makes it possible to find solutions for adjusting it, if it is not in accordance with the process step and/or end-product specifications. Thus, the idea of measuring, understanding, adjusting, monitoring and controlling variability throughout the production is a key topic of this course.

Quality in this course will be seen both in a wide context; i.e. as proof of product specification meaning that the variability of the food produced is known and below or within an acceptable limit to ensure customer satisfaction, as well as occasionally in a more narrow context; i.e. as food safety meaning that the amount of hazards is below a certain limit.

\subsection{Learning objectives}

The main objective of this course is to provide the students with knowledge on international food safety management system standards and food quality monitoring and verification tools. After completing the course, the students should be able to:

\subsubsection{Knowledge}

Describe how food safety is achieved, using HACCP, according to international standards on food safety management (e.g. ISO 22000). 
Show overview of how the above mentioned food safety management system standards can be applied in the food industry.
Describe how food production quality should be measured to ensure that process variation is under control.
Describe how to gain insight into a process using quality by design to be able to optimize a process.
 

\subsubsection{Skills} 

Apply ISO 22000 for food safety management.
Use SQC and sampling technologies to monitor and verify process and product specifications.
Communicate problems and solutions within food safety management.
Use monitoring systems (e.g. real-time measurements), Six Sigma and SQC like tools to get knowledge of and to monitor quality control points in a food production and also to understand how the measurement uncertainty of process steps will influence the uncertainty/variability of the food product parameters.
 

\subsubsection{Competences}  

Evaluate whether existing and/or new control strategies are appropriate in order to achieve safe and robust food products.
Evaluate how the raw materials and production steps can be monitored to get an even better control of the final food product.


\subsection{Teaching and learning methods}

Lectures, project work, and seminars. The lectures introduce theoretical and practical aspects of international systems on quality/food safety management and control. The project work and seminars will help the students to interpret and obtain an understanding of the above mentioned aspects.

\subsection{Exam}

The exam will be an ITX exam with a duration of 4 hours. The exam consists of multiple choice questions (50\% of the grade) and 1-2 essay questions (50\% of the grade) where the students will be asked to analyze and solve a problem related to the course content. The exam will be based on the course literature and the lectures.
